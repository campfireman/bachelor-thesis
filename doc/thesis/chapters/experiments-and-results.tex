\chapter{Experiments and Results}
\label{experiments-and-results}

Each experiment has a hypothesis, a test and a result.

\section{Hardware}
For the purpose of this thesis we had access to two machines. The smaller machine is a personal machine and a large machine rented at the provider Exoscale \cite{noauthor_exoscale_nodate}. They will be referred to as \textit{Balthazar} and \textit{Melchior} respectively. Their specifications are described in table \ref{hardware}.

\begin{table*}
    \begin{center}
        \begin{tabular}{ c|c|c }
            Component & Balthazar             & Melchior                \\
            \hline
            \hline
            CPU       & 6 Core AMD Ryzen 3600 & 24 Core Intel Broadwell \\
            RAM       & 32GB                  & 120GB                   \\
            GPU       & NVIDIA GTX 1660 Super & 3 $\cdot$ NVIDIA P100   \\
            VRAM      & 6GB                   & 3 $\cdot$ 16GB          \\
            Storage   & 500GB SSD             & 400GB SSD               \\
        \end{tabular}
    \end{center}
    \caption{Hardware specifications of the utilized machines}
    \label{hardware}
\end{table*}

\section{Parameters}
The behavior of the system is controlled by a multitude of parameters. The most relevant parameters are listed in table \ref{parameters}. If one of the parameters differs from the default values in an experiment, it will be mentioned. For each experiment, all parameters are pesisted by the system.

\begin{table}
    \begin{center}
        \begin{tabularx}{\textwidth}{ c|c|X }
            Name                       & Default & Explanation                                                                                   \\
            \hline
            \hline
            temp\_treshhold            & 60      & The number of moves for which the next move is sampled (cf. equation \ref{eq:move_selection}) \\
            update\_treshold           & 0.6     & The percentage of matches, that has to be won for a new network to be accepted                \\
            num\_MCTS\_sims            & 120     & The number of times the search tree is expanded during MCTS                                   \\
            num\_self\_play\_workers   & 14      & The number of workers used for parallel self-play                                             \\
            num\_arena\_workers        & 7       & The number of workers used for parallel Arena matches                                         \\
            load\_model                & true    & Indicates whether to load an existing model                                                   \\
            maxlen\_experience\_buffer & 1440000 & The maximum number of tuples ($s, \pi, z$) in the buffer                                      \\
            nnet\_size                 & mini    & The neural net used as introduced in \ref{neural_network_architecture}                        \\
            lr                         & 0.001   & The learning rate during training of the neural network                                       \\
            epochs                     & 10      & The number of of epochs during training of the neural network                                 \\
            batch\_size                & 64      & The size of the batches during training of the neural network                                 \\
        \end{tabularx}
    \end{center}
    \caption{The parameters of the training pipeline}
    \label{parameters}
\end{table}

\section{Validation}
\paragraph{Hypothesis} The modified framework still converges to optimal play for Othello and TicTacToe.
\paragraph{Test} Run pipeline with modified implemenation of Othello and TicTacToe from \cite{thakoor_suragnairalpha-zero-general_nodate} on Balthazar.
\paragraph{Result} Convergence to optimal play given for TicTacToe, for Othello it is very likely.

\section{Application}
