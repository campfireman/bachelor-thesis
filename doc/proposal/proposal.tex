\documentclass{../lib/llncs}
% Grundgröße 12pt, zweiseitig
% Standardpakete
% richtiges encoding fuer verschiedene compiler
\usepackage{iftex}
\ifPDFTeX
   \usepackage[utf8]{inputenc}
   \usepackage[T1]{fontenc}
   \usepackage{lmodern}
\else
   \ifXeTeX
     \usepackage{fontspec}
   \else
     \usepackage{luatextra}
   \fi
   \defaultfontfeatures{Ligatures=TeX}
\fi
% deutsche Silbentrennung
\usepackage[english]{babel}
\usepackage{amsmath}
\usepackage{cite}
\usepackage{float}
\usepackage{subfig}
\usepackage{changepage}

% Grafiken einbinden
\usepackage{graphicx}
\graphicspath{{../figures/}}

\usepackage{hyperref}
% tiefe des Inhaltsverzeichnisses
\setcounter{tocdepth}{2}

\usepackage{listings}
\usepackage{color}

\definecolor{dkgreen}{rgb}{0,0.6,0}
\definecolor{gray}{rgb}{0.5,0.5,0.5}
\definecolor{mauve}{rgb}{0.58,0,0.82}

\lstset{frame=tb,
  language=Python,
  aboveskip=3mm,
  belowskip=3mm,
  showstringspaces=false,
  columns=flexible,
  basicstyle={\small\ttfamily},
  numbers=none,
  numberstyle=\tiny\color{gray},
  keywordstyle=\color{blue},
  commentstyle=\color{dkgreen},
  stringstyle=\color{mauve},
  breaklines=true,
  breakatwhitespace=true,
  tabsize=3
}

\begin{document}

\title{Exploration of Abalone game-playing agents}
\author{Ture Claußen, 1531067, \email{ture.claussen@stud.hs-hannover.de}}
\authorrunning{T. Claußen}
\institute{Hochschule Hannover Fakultät IV}

{\def\addcontentsline#1#2#3{}\maketitle} % Wird gebraucht, damit der Title nicht im Inhaltsverzeichnis steht

\begin{abstract}
\end{abstract}


\section{Introduction}

\section{Background}
% 1)WHY
% 2)WHAT

\section{Related work}

\section{Aims and objectives}

\section{Methodology}
% 3) HOW

\subsection{Timeline}

\section{Preliminary table of contents}

% Literatur
\bibliographystyle{../lib/splncs04.bst}
\bibliography{../ref.bib}
\end{document}